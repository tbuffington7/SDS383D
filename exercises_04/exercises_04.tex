\documentclass[10pt]{article}
\usepackage{amsfonts}
\usepackage{fancyhdr}
\usepackage{comment}
\usepackage[letterpaper, top=2.5cm, bottom=2.5cm, left=2.2cm, right=2.2cm]%
{geometry}
\usepackage{amsmath}
\usepackage{subfig}
\usepackage{mathtools}
\usepackage{changepage}
\usepackage{enumitem}
\usepackage{amssymb}
\usepackage{graphicx}
\usepackage{hyperref}
\usepackage{listings}
\usepackage{color}
\usepackage{textcomp}
\usepackage{courier}

\definecolor{listinggray}{gray}{0.9}
\definecolor{lbcolor}{rgb}{0.96,0.96,0.96}
\lstset{
    backgroundcolor=\color{lbcolor},
    tabsize=4,,
    rulecolor=,
    language=Python,
        basicstyle=\footnotesize\ttfamily,
        upquote=true,
        aboveskip={1.0\baselineskip},
        columns=fixed,
        extendedchars=true,
        breaklines=true,
        prebreak = \raisebox{0ex}[0ex][0ex]{\ensuremath{\hookleftarrow}},
        frame=single,
        showtabs=false,
        showspaces=false,
        showstringspaces=false,
        identifierstyle=\ttfamily,
        keywordstyle=\color[rgb]{0,0,1},
        commentstyle=\color[rgb]{0.133,0.545,0.133},
        stringstyle=\color[rgb]{0.627,0.126,0.941},
}

\newcommand{\by}{\mathbf{y}}

\begin{document}

    \title{SDS 383D, Exercises 4: Hierarchical Models}
    \author{Tyler Buffington}
    \date{\today}
    \maketitle

    \section*{Math tests}
    \textbf{The data set in ``mathtest.csv'' shows the scores on a standardized math test from a sample of 10th-grade students at 100 different U.S.~urban high schools, all having enrollment of at least 400 10th-grade students.  (A lot of educational research involves ``survey tests'' of this sort, with tests administered to all students being the rare exception.)}
    
    \textbf{Let $\theta_i$ be the underlying mean test score for school $i$, and let $y_{ij}$ be the score for the $j$th student in school $i$.  Starting with the ``mathtest.R'' script, you'll notice that the extreme school-level averages $\bar{y}_i$ (both high and low) tend to be at schools where fewer students were sampled.}
    \begin{enumerate}[label=(\Alph*)]
    \item
    \textbf{Explain briefly why this would be.}
    
    When the sample sizes are small, the sample mean variance is large. This means that the sample means are more likely to be much higher or much lower than the "true" underlying mean score of the school. 
    \color{black}
    
    

    \item

    \end{enumerate}
\end{document}
\grid
